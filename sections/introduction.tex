\section{Introduction}

This project, entitled \emph{\textbf{Photocrypt}}, is basically a simple
computer program that implements the concept of \emph{steganography}. It can
be used to hide some secret information (messages) in a media file (image)
of sufficient size.

\subsection{Introduction to steganography}

Steganography is described as ``the art and science of writing hidden messages
in such a way that no one apart from the sender and the intended recipient
even knows that a message has been sent''. The word \emph{steganography} is
derived from Greek words \emph{steganos} (στεγανός), meaning ``covered,
concealed, or protected'', and \emph{graphein} (γράφειν), meaning ``writing''.

Digital steganography is the practice of concealing a message, image, or file
within another message, image, or file.

\subsection{Brief history of steganography}

Steganography as a whole has existed in many forms throughout much of history.
Some of their forms are given below:

\begin{itemize}
    \item{} The ancient Chinese wrote notes on small pieces of silk that they
        then wadded into little balls and coated in wax, to be swallowed by a
        messenger and retrieved at the messenger's gastro-intestinal
        convenience.
    \item{} Herodotus (485 -- 525 BC) recounts the story of Histaiaeus, who
        wanted to encourage Aristagoras of Miletus to revolt against the
        Persian king. In order to securely convey his plan, Histaiaeus shaved
        the head of his messenger, wrote the message on his scalp, and then
        waited for the hair to regrow. The messenger, apparently carrying
        nothing contentious, could travel freely. Arriving at his destination,
        he shaved his head and pointed it at the recipient.
    \item{} Giovanni Battista Porta described how to conceal a message within
        a hard-boiled egg by writing on the shell with a special ink made with
        an ounce of alum and a pint of vinegar. The solution penetrates the
        porous shell, leaving no visible trace, but the message stained on the
        surface of the hardened egg albumen, so it can be read when the shell
        is removed.
\end{itemize}
